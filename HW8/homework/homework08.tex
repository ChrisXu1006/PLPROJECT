\newif\ifbeamer\beamerfalse
\documentclass[10pt]{article}

\usepackage{../tex/jnf}
\usepackage{alltt}

\renewcommand{\labelenumi}{\textbf{(\alph{enumi})}}

\begin{document}

\bigredheader{Homework \#8}

\paragraph{Due} Wedneday, November 12, 2014 by 11:59pm 

\paragraph{Instructions} 
This assignment may be completed with one partner.  You and your
partner should submit a single solution on CMS. Please do not offer or
accept any other assistance on this assignment. Late submissions will
not be accepted.

\begin{exercise}
\begin{enumerate}
\item   Suppose that we add a new typing rule to the simply-typed
  $\lambda$-calculus, extended with integers and unit:
\begin{center}
\infrule[T-Funny]
{}
{ \Gamma \vdash (\lambda x\!:\!\typ{unit}.~x) : \typ{int} }
{}
\end{center} 
Does progress still hold?  Does preservation still hold? If so,
explain why (briefly). If not, give a counterexample.

\noindent \textbf{Answer:}

Progress does not hold. Counterexample: $(\lam{x\ty\typ{unit}}{x}) + 1$. This expression is well
typed. However, there is no evaluation rule that can apply to this expression to step. This expression
is not a value, either. So progress breaks.

Preservation still holds. We proved preservation using induction on the evaluation rules. This typing
rule does not affect evaluation rules, so we can still prove preservation.

\item Now suppose instead that we add a new evaluation rule to the
simply-typed $\lambda$-calculus:
\begin{center}
\infrule[Funny2]
{}
{ e_1 + e_2 \stepsone () }
{}
\end{center}
Does progress still hold?  Does preservation still hold? If so,
explain why (briefly). If not, give a counterexample.

\noindent \textbf{Answer:}

Progress still holds. This evaluation rule enriches the original evaluation rules. Additionaly, applying this rule
will step to (), which is a value. So combining this rule with original rules, the progress still holds.

Preservation does not hold. Counterexample: $n_1 + n_2 \stepsone ()$. Before the step, the expression had type 
\typ{int}. After the step, the type changes to \typ{uint}, so preservation breaks.

\end{enumerate}
\end{exercise}

\begin{exercise}
  The ``erasure'' of a System F term into a pure $\lambda$-calculus
  term can be defined as follows:
%
\[
\begin{array}{rcl}
\mathit{erase}(x) & = & x\\
\mathit{erase}(\lam{x \ty \tau}{e}) & = & \lam{x}{\mathit{erase}(e)}\\
\mathit{erase}(e_1~e_2) & = & (\mathit{erase}(e_1)~ \mathit{erase}(e_2))\\
\mathit{erase}(\Lam{X}{e}) & = & \lam{z}{\mathit{erase}(e)}~\qquad \text{where}~z~\text{fresh}\\
\mathit{erase}(e~[\tau]) & =& \mathit{erase}(e)~\lam{z}{z}
\end{array}
\]
%
Prove that if $e \arrow e'$ then
$\mathit{erase}(e) \Rightarrow \mathit{erase}(e')$ by induction (where
to avoid any confusion we let ``$\arrow$'' stand for the evaluation
relation in System F and ``$\Rightarrow$'' stand for the evaluation
relation for the pure $\lambda$-calculus). Although this property is
``obvious,'' proving it rigorously is still good practice!

\noindent \textbf{Answer:}

We prove this property by induction on $e$:
\begin{itemize}
	\item \textbf{Case} $e = x$: 
	
	this case is impossible, since it's not a value and cannot step.
	\item \textbf{Case} $e = \lam{x\ty\tau}{e}$: 
	
	this is already a value, it cannot step further. Trivilly Proved.
	\item \textbf{Case} $e = e_1e_2$: 
	
	if $e_1$ is not a value, then by \rulename{Context}, we have $e_1e_2 \arrow e_1'e_2$ and 
	$e_1 \arrow e_1'$. Hence, $e' = e_1'e_2$ Using the induction hypothesis, 
	we have $erase(e_1) \Rightarrow erase(e_1')$.
	Therefore, $erase(e) = erase(e_1e_2) = (erase(e_1)erase(e_2)) \Rightarrow (erase(e_1')erase(e_2))
	= erase(e_1'e_2) = erase(e')$.
	
	if $e_1$ is a value and $e_2$ is not a value, then by \rulename{Context}, we have $e_1e_2 \arrow e_1e_2'$ and 
	$e_2 \arrow e_2'$. Hence, $e' = e_1e_2'$ Using the induction hypothesis, 
	we have $erase(e_2) \Rightarrow erase(e_2')$.
	Therefore, $erase(e) = erase(e_1e_2) = (erase(e_1)erase(e_2)) \Rightarrow (erase(e_1)erase(e_2'))
	= erase(e_1e_2') = erase(e')$.
	
	if both $e_1$ and $e_2$ are values, then $e_1 = \lam{x}{e^*}$ and $e_2 = v$. Using \rulename{$\beta$-Reduction},
	we have $e' = e^*\{v/x\}$. So we have $erase(e) = erase((\lam{x}{e^*})v) = erase(\lam{x}{e^*})erase(v) =
	(\lam{x}{erase(e^*)})erase(v) \Rightarrow$ 
	
	$erase(e^*)\{erase(v)/x\} = erase(e^*\{v/x\}) = erase(e') $
	\item \textbf{Case} $e = \Lam{X}{e}$:

	this is already a value, it cannot step further. Trivilly Proved.
	\item \textbf{Case} $e^*[\tau]$:

	$e^*$ must be $\Lam{X}{e_1}$. So $e = \Lam{X}{e_1}[\tau]$, and $e' = e_1\{\tau/X\}$.
	We have $erase(e) = erase(\Lam{X}{e_1})\lam{z}{z} = (\lam{z}{erase(e_1)})\lam{z}{z} \Rightarrow erase(e_1)$ in
	which $z$ is a fresh variable of $e_1$. 
	We also have $erase(e') = erase(e_1\{\tau/X\}) = erase(e_1)$. 
	The reason for $erase(e_1\{\tau/X\}) = erase(e_1)$ is that
	no matter whether $\tau$ replaces $X$ in $e_1$ or not, it will be erased anyway. As a result, we can derive
	$erase(e) \Rightarrow erase(e')$.
\end{itemize}
\end{exercise}

\begin{exercise}
Consider System F extended with integers, booleans, and sums:
\begin{align*}
e    &::= x \mid \lam{x\ty\tau}{e} \mid e_1~e_2 \mid \Lam{X}{e} \mid e~[\tau] \mid n \mid e_1 + e_2  \\
     & \qquad \mid \lamfnt{true}  \mid \lamfnt{false} \mid e_1 = e_2 \mid \cond[\lamfnt]{e_1}{e_2}{e_3}  \\
     & \qquad \mid \inleft{\tau_1}{\tau_2}{e} \mid \inright{\tau_1}{\tau_2}{e} \mid \sumcase{e_1}{e_2}{e_3}\\
\tau &::= \typ{int} \mid \typ{bool} \mid \tau_1 \arrow \tau_2 \mid X \mid \forall X.~\tau \mid \tau_1 + \tau_2
\end{align*}
%
Two expressions $e_1$ and $e_2$ are said to be \emph{behaviorally
equivalent} if for any expression $f$, the expression $f~e_1$ behaves
the same as $f~e_2$. That is, it is impossible for $f$ to distinguish
$e_1$ from $e_2$. For example, the expressions
$\lam{x\ty\typ{unit}}{4+9}$ and $\lam{x\ty\typ{unit}}{13}$ are
behaviorally equivalent, since any function $f$, say
$\lam{t\ty\typ{unit}\arrow\typ{int}}{29+(t~())}$, behaves exactly the
same when given either one. However, the expressions
$\lam{x\ty\typ{int}}{0}$ and
$\lam{x\ty\typ{int}}{\cond[\lamfnt]{x=42}{1}{0}}$ are not behaviorally
equivalent because the expression
$\lam{y\ty{\typ{int}\arrow\typ{int}}}{y~42}$ distinguishes them.

For each of the following types, list the behaviorally \emph{distinct}
expressions of that type. That is, any expression that has that type
should be behaviorally equivalent to one of the expressions in your
list.  (Recall that all System F expressions terminate so you do not
need to consider non-termination as a possible behavior.)
%
\begin{enumerate}
\begin{minipage}{.5\textwidth}
\item $\forall A.~ A \rightarrow A$\\
\item $\forall A.~ A \rightarrow A \rightarrow A$\\
\item $\forall A.~\forall B.~ A \rightarrow (A \rightarrow B) \rightarrow B$\\
\item $\forall A.~\forall B.~ A \rightarrow B \rightarrow A + B$\\
\end{minipage}\begin{minipage}{.5\textwidth}
\item $\forall A.~ A$\\
\item $\forall A.~ A \rightarrow \typ{bool} \rightarrow A$\\
\item $\forall A.~ A \rightarrow A \rightarrow \typ{bool} \rightarrow A$\\
\bigskip\bigskip
\end{minipage}
\end{enumerate}

\noindent \textbf{Answer:}
\begin{enumerate}
\item $\forall A.~ A \rightarrow A:$ 

$\Lam{A}{\lam{x\ty A}}{x}$
\item $\forall A.~ A \rightarrow A \rightarrow A:$ 

$\Lam{A}{\lam{x\ty A}{\lam{y\ty A}{x}}}$ 

$\Lam{A}{\lam{x\ty A}{\lam{y\ty B}{y}}}$
\item $\forall A.~ \forall B.~ A\rightarrow (A \rightarrow B) \rightarrow B:$ 

$\Lam{A}{\Lam{B}{\lam{x\ty A}{\lam{f\ty (A \rightarrow B)}{f~x}}}}$
\item $\forall A.~\forall B.~ A \rightarrow B \rightarrow A + B: $

$\Lam{A}{\Lam{B}{\lam{x\ty A}{\lam{y\ty B}{\inleft{A}{B}{x}}}}},$ 

$\Lam{A}{\Lam{B}{\lam{x\ty A}{\lam{y\ty B}{\inright{A}{B}{y}}}}}$
\item $\forall A.~A:$ This type is the void type, therefore, there is no behaviorally distinct expressions of that type
\item $\forall A.~ A \rightarrow \typ{bool} \rightarrow A: $

$\Lam{A}{\lam{x\ty A}{\lam{b\ty\typ{bool}}{x}}}$
\item $\forall A.~ A \rightarrow A \rightarrow \typ{bool} \rightarrow A: $

$\Lam{A}{\lam{x\ty A}{\lam{y\ty A}{\lam{z\ty \typ{bool}}{x}}}},$

$\Lam{A}{\lam{x\ty A}{\lam{y\ty A}{\lam{z\ty \typ{bool}}{y}}}},$

$\Lam{A}{\lam{x\ty A}{\lam{y\ty A}{\lam{z\ty \typ{bool}}{\cond[\lamfnt]{z}{x}{y}}}}},$

$\Lam{A}{\lam{x\ty A}{\lam{y\ty A}{\lam{z\ty \typ{bool}}{\cond[\lamfnt]{z}{y}{x}}}}},$
\end{enumerate}
\end{exercise}

\begin{exercise}
  For each of the following, write ``Yes'' if it is okay to allow
  these types to be in the subtype relation or ``No'' if not. In
  addition, if your answer is ``No'' give a counterexample that shows
  how type soundness would break.
\begin{itemize}
\item $\typ{int} \subty \typ{unit}$
\item $\{ l : \top \} \subty \{ l : \typ{bool} \}$
\item $\{ \} \subty \{ x : \top \}$ 
\item $(\top \times \{ x \ty \typ{unit} \}) \subty (\{\} \times \top)$
\item $(\{ x \ty \typ{int} \} \arrow \typ{int}) \subty (\{ x \ty \typ{int}, y \ty \typ{int} \} \arrow \typ{int})$
\item $(\{ x \ty \typ{int}, y \ty \typ{int} \} \arrow \typ{int}) \subty (\{ x \ty \typ{int} \} \arrow \typ{int})$
\end{itemize}
\noindent \textbf{Answer:}
\begin{itemize}
\item $\typ{int} \subty \typ{unit}:$ Yes.
\item $\{ l : \top \} \subty \{ l : \typ{bool} \}: \ $ No. Counterexample, ($\lam{x\ty \{l\ty\typ{bool}\}}{\cond[\lamfnt]{x.l}{e_1}{e_2}} \ \ true$). If bool is replaced with $\top$ type, the expression cannot be progressed.
\item $\{ \} \subty \{ x : \top \}: \ $ No. Counterexample is $\lam{y\ty\{x\ty\top\}}{y.x}$
\item $(\top \times \{ x \ty \typ{unit} \}) \subty (\{\} \times \top): \ $ Yes.
\item $(\{ x \ty \typ{int} \} \arrow \typ{int}) \subty (\{ x \ty \typ{int}, y \ty \typ{int} \} \arrow \typ{int}): \ $ No. Counterexample: $\lam{r\ty \{ x \ty \typ{int}, y \ty \typ{int} \}}{(r.y)}$
\item $(\{ x \ty \typ{int}, y \ty \typ{int} \} \arrow \typ{int}) \subty (\{ x \ty \typ{int} \} \arrow \typ{int}): \ $ Yes.
\end{itemize}
\end{exercise}

\begin{exercise}
  Consider the simply-typed $\lambda$-calculus with records and
  subtyping.
%
\[
\begin{array}{rcl}
\tau & ::= & \{ l_1 \ty \tau_1, \dots, l_n \ty \tau_n \} \mid \tau_1 \arrow \tau_2\\
e    & ::= & x \mid e_1~e_2 \mid \lambda x \ty\tau .~ e \mid \{ l_1 = e_1 , \dots, l_n = e_n \} \mid e.l\\
\end{array}
\]
%
Prove progress and preservation. \\

\noindent \textbf{Proof}\\[0.1cm]
\noindent \textbf{Preservation.}
\begin{itemize}
\item For the evaluation rule to access the field of a location in the records,\\ here $e = \{ l_1 \ty \tau_1, \dots, l_n \ty \tau_n \}.l_i \ and \ e' = v_i$. Since e is well typed, so we can get $\Gamma \vdash e: \tau_i$ and $\Gamma \vdash v_i: \tau_i$. Therefore, we have the preservation hold for this evaluation rule.
\end{itemize}
\noindent \textbf{Progress.}
\begin{itemize}
\item For the first type rules of records
\begin{center}
\infrule[T-Record1]
{\forall. i \in 1 \dots n, \ \ \Gamma \vdash e_i : \tau_i}
{ \Gamma \vdash \{ l_1 = e_1, \dots, l_n = e_n \} : \{ l_1 \ty \tau_1, \dots, l_n \ty \tau_n \}  }
{}
\end{center}
From the induction hypothesis, for $i \in \{1 \dots n\},$ either $e_i$ is a value or there is an $e_i'$ such that $e_i \rightarrow e_i'$. If $e_i$ is not a value and for all $j \in \{1 \dots n-1\}$ $e_j$ is a value. By CONTEXT, we can get $\{ l_1 = v_1, \dots, l_i = e_i, \dots, l_n = e_n \} \rightarrow \{ l_1 = v_1, \dots, l_i = e_i', \dots, l_n = e_n \}$. If all $e_i$ are values, then e is already a value.
\item For the second type rules of records
\begin{center}
\infrule[T-Record2]
{\Gamma \vdash \{l_1 \ty \tau_1, \dots, l_n \ty \tau_n\}}
{ \Gamma \vdash e.l_i \ty \tau_i }
{}
\end{center}
Here e is $e_1.l_i$ From the induction hypothesis, $e_1$ is either a value or there is a $e_1'$ such that $e_1 \rightarrow e_1'$. If $e_1$ is not a value, by CONTEXT, we have $e_1.l_i \rightarrow e_1'.l_i$ If $e_1$ is a value, by the evaluation of records to access the field of a location, we have $e_1.l_i \rightarrow v_i$
\item SUBSUMPTION. Here e is $e \ty \tau'$. From the induction hypothesis, we have that $e \ty \tau$ is either a value or there is a e' such that $e \rightarrow e'$. Therefore, the progress of SUBSUMPTION is proved.
\end{itemize}
\end{exercise}

\begin{debriefing} \hfill\\[-4ex]
\begin{enumerate*}
\item How many hours did you spend on this assignment?\\[0.2cm]
\noindent \textbf{Answer:} 4 hours\\

\item Would you rate it as easy, moderate, or difficult?\\[0.2cm]
\noindent \textbf{Answer:} Moderate\\

\item Did everyone in your study group participate?\\[0.2cm]
\noindent \textbf{Answer:} Yes\\

\item How deeply do you feel you understand the material it covers (0\%--100\%)?\\[0.2cm]
\noindent \textbf{Answer:} About 85\% \\

\item If you have any other comments, we would like to hear them!
  Please send email to \texttt{jnfoster@cs.cornell.edu}.
\end{enumerate*}
\end{debriefing}

\end{document}

