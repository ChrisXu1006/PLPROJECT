\newif\ifbeamer\beamerfalse
\documentclass[11pt]{article}

\usepackage{../tex/jnf}
\usepackage{../tex/angle}

\renewcommand{\labelenumi}{\textbf{(\alph{enumi})}}
\solutiontrue
\begin{document}

\bigredheader{Homework \#6}

\vspace*{-1.25\bigskipamount}

\paragraph{Due} 
%
Wednesday, October 29, 2014 at 11:59pm.

\paragraph{Instructions} 
%
This assignment may be completed with one partner.  You and your
partner should submit a single solution on CMS. Please do not offer or
accept any other assistance on this assignment. Late submissions will
not be accepted.

\begin{exercise}
Recall the syntax of the untyped $\lambda$-calculus. 
\[
\begin{array}{rcl}
e & ::= & x \mid \lam{x}{e} \mid e_1~e_2\\
v & ::= & \lam{x}{e}
\end{array}
\]
Define a large-step semantics for this language under call-by-name
(lazy) evaluation.
\newline
\newline
\textbf{Answer:}
\begin{center}
\infrule
{}{\lam{x}{e} \stepsto \lam{x}{e}}{}
\qquad
\infrule
{e_1 \stepsto \lam{x}{e} \quad e[e_2/x] \stepsto \lam{x}{e'}}{e_1 e_2 \stepsto \lam{x}{e'}}{}
\end{center}

\end{exercise}

\begin{exercise}
In this exercise you will develop encodings in call-by-value untyped
$\lambda$-calculus.
\begin{enumerate}
\item Define an expression $\nm{EQUAL}$ that tests whether two Church numerals 
  are equal, and returns the corresponding Church boolean.\\

\noindent \textbf{Answer:}
\begin{center}
	$\nm{MINUS} \triangleq \lam{x}{\lam{y}{x \ \ \nm{PRED} \ \ y}}$\\
\end{center}
\noindent Actually, Church Numerals can only encode natural numbers. Consequently, when we apply $\nm{ISZERO}$ operator for $\nm{MINUS} \ \ m \ \ n$ where m is less or equal to n, the result will always be \nm{TRUE}. So we get the definition of $\nm{LEQ}$ as following:
\begin{center}
	$\nm{LEQ} \ \triangleq \lam{x}{\lam{y}{\nm{ISZERO} \ \ (\nm{MINUS} \ \ x \ \ y)}}$\\
\end{center}
\noindent Making use of the fact that $x \leq y \ and \ y \leq x$ implies $x = y$, we can obtain the definition of \nm{EQUAL}:
\begin{center}
	$\nm{EQUAL} \triangleq \lam{x}{\lam{y}{\nm{AND} \ \ (\nm{LEQ} \ \ x \ \ y) \ \ (\nm{LEQ} \ \ y \ \ x)}}$
\end{center}
\item Define expressions for each of the
  following list constructs: $\nm{NIL}$, $\nm{CONS}$, $\nm{HD}$,
  $\nm{TL}$, and $\nm{IS\_NIL}$. To keep things simple, applying
  either $\nm{HD}$ or $\nm{TL}$ to $\nm{NIL}$ should produce
  $\nm{NIL}$. Also, if you find it helpful, you may use the encodings
  of booleans and pairs discussed in class.\\

\noindent \textbf{Answer:}
\begin{center}
	$\nm{NIL} = \lam{L}{\nm{TRUE}}$\\
	$\nm{CONS} = \lam{H}{\lam{T}{(\lam{L}{L~H~T})}}$\\
	$\nm{HD} = \lam{L}{\nm{IF}~(\nm{IS\_NIL}~L)~L~(L~\nm{TRUE})}$\\
	$\nm{TL} = \lam{L}{\nm{IF}~(\nm{IS\_NIL}~L)~L~(L~\nm{FALSE})}$\\
	$\nm{IS\_NIL} = \lam{L}{L(\lam{H}{\lam{T}{\nm{FALSE}}})}$
\end{center}
\item Define a function $\nm{SUM}$ that adds up an encoded list of encoded Church
  numerals.\\

\noindent \textbf{Answer:}
\begin{center}
	$\nm{SUM'} \triangleq \lam{f}\lam{L}{\nm{IF}~(\nm{IS\_NIL}~L)~\nm{ZERO}~(\nm{PLUS}~(\nm{HD}~L)~(f~f~(\nm{TL}~L)))}$\\
	$\nm{SUM} \triangleq \nm{SUM'}~\nm{SUM'}$
\end{center}
\end{enumerate}
\end{exercise}

\begin{exercise}
The simply-typed $\lambda$-calculus is not Turing complete---as shown
in lecture, every well-typed expression reduces to a value. It is not
hard to recover Turing completeness by adding an explicit fixpoint
operator to the language. We first extend the syntax of the language
with a new construct,
\[
e ::= \dots \mid \typ{fix}~e\\
\]
and then add evaluation contexts and a small-step semantics rule to
handle it:
\begin{center}
$E ::= \dots \mid \typ{fix}~E$

\infrule{}{\typ{fix}~(\lam{x\ty\tau}{e}) \stepsone e\{\typ{fix}~(\lam{x\ty\tau}{e}) / x\}}{} 
\end{center}

\begin{enumerate}
\item Give the typing rule for fixpoints.
\newline
\newline
\textbf{Answer:}
\newline
\begin{center}
\infrule[T-FIX]{\Gamma \vdash e\ty\tau \stepsone \tau}{\Gamma \vdash fix\ e\ty\tau}{}
\end{center}
\item Give the case of the progress lemma for fixpoints, assuming the proof is by induction on typing derivations.
\newline
\newline
\textbf{Answer:}
	\begin{itemize}
		\item T-FIX \newline
		Here $e \equiv \typ{fix}\ e'$ and $e':\tau \stepsone \tau$. By the inductive hypothesis, either $e'$ is a value
		or there is an $e''$ such that $e' \stepsone e''$.
		\newline
		If $e'$ is not a value, then by \rulename{Context}, $\typ{fix}\ e' \stepsone \typ{fix}\ e''$. If $e'$ is a
		value, by inversion, $e'$ must be an abstraction, so $e' \equiv \lam{x\ty\tau}{e^{\ast}}$. 
		Use the evaluation rule for $\typ{fix}\ e$, we have $\typ{fix}~(\lam{x\ty\tau}{e^{\ast}}) \stepsone
		 e^{\ast}\{\typ{fix}~(\lam{x\ty\tau}{e^{\ast}}) / x\}$
	\end{itemize}
\item Give the case of the preservation lemma for fixpoints, assuming the proof is done by induction on evaluation rules.
\newline
\newline
\textbf{Answer:}
	\begin{itemize}
		\item \rulename{Fix} \newline
		Here $e \equiv \typ{fix}\ (\lam{x\ty\tau}{e^{\ast}})$ and $e' \equiv e^{\ast}\{\typ{fix}\ 
		(\lam{x\ty\tau}{e^{\ast}})/x\}$. Since $e$ is well typed, we have derivations showing that
		$\Gamma \vdash \typ{fix}\ (\lam{x\ty\tau}{e^{\ast}})\ty\tau$. There is only one typing rule for fixpoint, 
		\rulename{T-Fix}, from which we know $\Gamma \vdash \lam{x\ty\tau}{e^{\ast}}\ty\tau \stepsone \tau$.
		There is only one typing rule for abstractions,
		\rulename{T-Abs}, from which we know $\Gamma,x\ty\tau \vdash e^{\ast}\ty\tau$. By the substitution lemma,
		we have $\vdash e^{\ast}\{\typ{fix}\ (\lam{x\ty\tau}{e^{\ast}})/x\}:\tau$ as required. 
	\end{itemize}
\end{enumerate}

\end{exercise}



\begin{debriefing} \hfill\\[-4ex]
\begin{enumerate*}
\item How many hours did you spend on this assignment?\\[0.5mm]
\noindent \textbf{Answer:} 10 hours
\item Would you rate it as easy, moderate, or difficult?\\[0.5mm]
\noindent \textbf{Answer:} Difficult
\item Did everyone in your study group participate?\\[0.5mm]
\noindent \textbf{Answer:} Yes 
\item How deeply do you feel you understand the material it covers (0\%--100\%)?\\[0.5mm]
\noindent \textbf{Answer:} About 80\%
\item If you have any other comments, we would like to hear them!
  Please write them here or send email to
  \mtt{jnfoster@cs.cornell.edu}.
\end{enumerate*}
\end{debriefing}

\end{document}

