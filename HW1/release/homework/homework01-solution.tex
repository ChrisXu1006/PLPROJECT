\documentclass[11pt]{article}

\usepackage{../tex/jnf}
\solutionfalse

\renewcommand{\labelenumi}{\textbf{(\alph{enumi})}}

\begin{document}

\bigredheader{Homework \#1}

\vspace*{-1.25\bigskipamount}

%% \paragraph{Changelog}
%% \begin{itemize}
%% \item Version 2 (this version): fixed typo; the expected output for the
%%   conditional is $7$ not $42$.
%% \item Version 1: Initial version.
%% \end{itemize}

\paragraph{By.} 
%
\textbf{Rui Xu(rx37), Yao Wang(yw438) \hfill \today}

\paragraph{Instructions.} 
%
This assignment may be completed with one partner. You and your
partner should submit a single solution on CMS. Please do not offer or
accept any other assistance on this assignment. Late submissions will
not be accepted.

\newcommand{\ELVIS}[2]{#1~\mtt{?:}~#2}

\begin{exercise}
  Suppose we extend our language of arithmetic expressions with an
  ``Elvis'' operator that evaluates to its first sub-expression if it
  is non-zero, and its second sub-expression otherwise:
%
\[
e ::= \VAR{x} 
 \mid \LIT{n} 
 \mid \PLUS{e_1}{e_2} 
 \mid \TIMES{e_1}{e_2} 
 \mid \ASSGN{x}{e_1}{e_2}
 \mid \lowershade{$\ELVIS{e_1}{e_2}$}
\]
%
The following inference rules extend the operational semantics to
handle these expressions:
%
\begin{mathpar}
\inferrule*[right=Elvis1]
{ \CONFIG{\sigma}{e_1} \stepsone \CONFIG{\sigma'}{e_1'} }
{ \CONFIG{\sigma}{\ELVIS{e_1}{n_2}} \stepsone \CONFIG{\sigma'}{\ELVIS{e_1'}{n_2}} }

\inferrule*[right=Elvis2]
{ \CONFIG{\sigma}{e_2} \stepsone \CONFIG{\sigma'}{e_2'} }
{ \CONFIG{\sigma}{\ELVIS{e_1}{e_2}} \stepsone \CONFIG{\sigma'}{\ELVIS{e_1}{e_2'}} }

\\
\inferrule*[right=Elvis3]
{ }
{ \CONFIG{\sigma}{\ELVIS{0}{n}} \stepsone \CONFIG{\sigma}{n} }

\inferrule*[right=Elvis4]
{ m \neq 0}
{ \CONFIG{\sigma}{\ELVIS{m}{n}} \stepsone \CONFIG{\sigma}{m} }
\end{mathpar}
%
Unfortunately, they do not implement the intended semantics. For
example, evaluating
\[
\ASSGN{x}{\PLUS{1}{1}}{(\ELVIS{x}{\ASSGN{x}{\PLUS{1}{x}}{0}})}
\]
yields $3$, not $2$.

\begin{enumerate*}
\item Give the derivation tree for the first step of evaluation for
  the above expression using $\sigma$.\\[0.1cm]
\noindent \textbf{Answers:}
\begin{mathpar}
	\inferrule*[right=Assign1]
	{
		\inferrule*[right=Add]
		{$1 + 1 = 2$}
		{\CONFIG{\sigma}{\ $x := 1 + 1$} \stepsone \CONFIG{\sigma}{\ $x := 2$}}
	}
	{\CONFIG{\sigma}{\ \ASSGN{$x$}{$1 + 1$}{\ELVIS{$x$}{\ \ASSGN{$x$}{$1 + x$}{$0$}}}} \stepsone \CONFIG{\sigma}{\ASSGN{$x$}{$2$}{\ELVIS{$x$}{\ASSGN{$x$}{$1 + x$}{$0$}}}}}
\end{mathpar}
\item Give the rest of the sequence of configurations encountered
  during the evaluation of the above expression. Note that you do not
  need to write out the derivation trees. Just give the sequence of
  configurations encountered during evaluation.\\[0.1cm]
\noindent \textbf{Answer:}
\begin{tabbing}
\CONFIG{\sigma}{\ \ASSGN{$x$}{$2$}{(\ELVIS{$x$}{\ASSGN{$x$}{$1 + x$}{$0$}})}}\quad\= \stepsone \= \quad\CONFIG{\sigma[$x$ \ \mapsto \ $2$]}{\ \ELVIS{$x$}{\ASSGN{$x$}{\ $1 + x$}{$0$}}}\\ 
																				  \> \stepsone \> \quad\CONFIG{\sigma[$x$ \ \mapsto \ $2$]}{\ \ELVIS{$x$}{\ASSGN{$x$}{\ $1 + 2$}{$0$}}}\\
																				  \> \stepsone \> \quad\CONFIG{\sigma[$x$ \ \mapsto \ $2$]}{\ \ELVIS{$x$}{\ASSGN{$x$}{\ $3$}{$0$}}}\\
																				  \> \stepsone \> \quad\CONFIG{\sigma[$x$ \ \mapsto \ $3$]}{\ \ELVIS{$x$}{$0$}}\\
																				  \> \stepsone \> \quad\CONFIG{\sigma[$x$ \ \mapsto \ $3$]}{\ \ELVIS{$3$}{$0$}}\\
																				  \> \stepsone \> \quad\CONFIG{\sigma[$x$ \ \mapsto \ $3$]}{\ $3$}
\end{tabbing}
\item Revise the rules so they implement the intended semantics. Using
  your rules, evaluating the above expression with the empty store
  should evaluate to $2$ (you do not need to show this).\\[0.1cm]
\noindent \textbf{Answer:} \\
The solution is removing all rules(ELVIS 1,2,3,4), and define the following new rules:
\begin{mathpar}
	\inferrule*[right=ELVIS1\_new]
	{\CONFIG{\sigma}{e_1} \stepsone \CONFIG{\sigma'}{e_1'}}
	{\CONFIG{\sigma}{\ELVIS{e_1}{e_2}} \stepsone \CONFIG{\sigma'}{\ELVIS{e_1'}{e_2}}}
	
	\inferrule*[right=ELVIS2\_new]
	{m \neq 0}
	{\CONFIG{\sigma}{\ELVIS{m}{e_2}} \stepsone \CONFIG{\sigma}{m}}
	
	\inferrule*[right=ELVIS3\_new]
	{ }
	{\CONFIG{\sigma}{\ELVIS{0}{e_2}} \stepsone \CONFIG{\sigma}{e_2}}
\end{mathpar}
\end{enumerate*}
\end{exercise}

\begin{exercise}
  Now suppose instead we extend our language with truncating integer
  division:
%
\[
e ::= \VAR{x} 
 \mid \LIT{n} 
 \mid \PLUS{e_1}{e_2} 
 \mid \TIMES{e_1}{e_2} 
 \mid \lowershade{$\DIV{e_1}{e_2}$}
 \mid \ASSGN{x}{e_1}{e_2}
\]
%
We add new evaluation rules to the operational semantics to handle
these expressions:
%
\begin{mathpar}
\inferrule*[right=LDiv]
{ \CONFIG{\sigma}{e_1} \stepsone \CONFIG{\sigma'}{e_1'} }
{ \CONFIG{\sigma}{\DIV{e_1}{e_2}} \stepsone \CONFIG{\sigma'}{\DIV{e_1'}{e_2}} }

\inferrule*[right=RDiv]
{ \CONFIG{\sigma}{e_2} \stepsone \CONFIG{\sigma'}{e_2'} }
{ \CONFIG{\sigma}{\DIV{e_1}{e_2}} \stepsone \CONFIG{\sigma'}{\DIV{e_1}{e_2'}} }

\inferrule*[right=Div]
{ n \neq 0 \\ p = m / n }
{ \CONFIG{\sigma}{\DIV{m}{n}} \stepsone \CONFIG{\sigma}{p} }
\end{mathpar}

\begin{enumerate*}
\item Is the semantics deterministic? If not, give a counter example.\\[0.1cm]
\noindent \textbf{Answer:} The semantics is not deterministic.\\
\noindent The counter example is \CONFIG{\sigma}{16/8/4}. When we apply LDIV rule, the configuration can step into \CONFIG{\sigma}{2/4}. On the other hand, if we use RDIV rule, the configuration can also be transformed to \CONFIG{\sigma}{16/2}. And these two configurations are absolutely different.\\
\item Does the semantics terminate? If not, give a counterexample.\\[0.1cm]
\noindent \textbf{Answer:} The semantics terminates.\\
\item Does the progress theorem still hold? If not, give a counterexample.\\[0.1cm]
\noindent \textbf{Answer:} The progress theorem doesn't hold for truncating integer division. If the denominator of evaluation is zero(like $e_1$ = 2/0), the current configuration cannot progress to any other configuration, while the expression itself is not integer.  
\end{enumerate*}
\end{exercise}

\begin{exercise}
  Let $c$ range over characters drawn from a fixed alphabet.  The
  following grammar describes the set of strings over the same
  alphabet.
%
\[
s ::= \epsilon \mid c \cdot s
\]
%
There are two cases: $\epsilon$ represents the empty string and $c
\cdot s$ represents the string obtained by concatenating $c$ and $s$.
%
\begin{enumerate*}
\item Give an inductive definition of the set of strings $s$ using
  inference rules.\\[0.1cm]
\noindent \textbf{Answer:}\\
\begin{mathpar}
	\inferrule*
	{ }
	{ \quad \epsilon \in s \quad}
	
	\inferrule*
	{ c \in defined \ alphabet, \ s \in s }
	{ \quad\quad c \cdot s \in s \quad\quad }
\end{mathpar}

\item Give the cases we would need to establish to prove that a
  property $P$ holds of all strings $s$ by structural
  induction---e.g., $P(\epsilon)$, etc.\\[0.1cm]
\noindent \textbf{Anwers:}\\
\noindent \textbf{Base cases:} $P(\epsilon)$\\
\noindent \textbf{Induction cases:} $®P(s) \Rightarrow P(c \cdot s)$\\

\item Complete the definition of the $\mathit{length}$ relation which
  associates a string to the number of characters contained in it, by
  giving the cases for single characters and concantenations.
\begin{mathpar}
\inferrule*
{ }
{ (\epsilon, 0) \in \mathit{length}}
\end{mathpar}
\end{enumerate*}
\end{exercise}
\quad\quad\quad \textbf{Answers:}\\
\begin{mathpar}
	\inferrule*[right=concantenations]
	{ c \in defined \ alphabet \ \ (s, m) \in \mathit{length}}
	{(c \cdot s, m+1) \in \mathit{length}}
\end{mathpar}
\begin{exercise}
Explain the flaw in the following ``proof.''

\begin{description}
\item \textbf{Theorem.} If $n$ is an even number greater than or equal
  to $2$, then there exists $i$ such that $n = 2^i$.

\item \textbf{Proof.} We will show that $P(n)$ holds for all $n$ by
  mathematical induction.

\begin{itemize}
\item \textbf{Base case:} $P(0)$ holds since $0$ is not greater than
  or equal to $2$. 

\item \textbf{Induction case:} We must show that $P(n)$ implies
  $P(n+1)$. We analyze several sub-cases. If $n$ is $1$ then $P(1)$
  holds since $1 = 2^0$. If $n$ is $2$ then $P(2)$ holds since $2 =
  2^1$. Otherwise, if $n$ is odd, then the assumption that $n$ is even
  is false and $P(n)$ vacuously holds. Otherwise, $n$ is even, so
  there exists an $m$ such that $n = 2 \times m$. By induction
  hypothesis, there exists $j$ such that $m = 2^j$. Hence, we have,
\[ 
\begin{array}{rcll}
n & = & 2 \times m   & \text{As $n$ is even}\\
  & = & 2 \times 2^j & \text{By induction hypothesis on $m$}\\
  & = & 2^{j+1}       & \text{By substitution}
\end{array}
\]
 which finishes the sub-case and the inductive proof. \hfill$\Box$
\end{itemize}
\end{description}
\noindent \textbf{Answer:}
\begin{itemize}	
\item In terms of Induction case, induction hypothesis only implies that there exists $j$ 	such that $m = 2^j$ for even numbers which are greater or equal to 2 rather than all integers greater or equal to 2. Therefore, if $m$ is odd, the step from the $ n = 2 \times m $ to $ n = 2 \times 2^j $ doesn't 	hold. 
\end{itemize}
\end{exercise}

\begin{debriefing} \hfill\\[-4ex]
\begin{enumerate*}
\item How many hours did you spend on this assignment?\\
\noindent \textbf{Answer:} We spent about two hours on this assignment.\\
\item Would you rate it as easy, moderate, or difficult?\\
\noindent \textbf{Answer:} The difficulty of the assignment is moderate.\\ 
\item How deeply do you feel you understand the material it covers (0\%–100\%)?\\
\noindent \textbf{Answer:} About 90\%.\\
\item If you have any other comments, we would like to hear them!
  Please write them here or send email to
  \mtt{jnfoster@cs.cornell.edu}.
\end{enumerate*}
\end{debriefing}
\end{document}

