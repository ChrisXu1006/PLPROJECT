\newif\ifbeamer\beamerfalse

\documentclass[11pt]{article}

\usepackage{../tex/jnf}
\usepackage{../tex/angle}

\solutionfalse

\renewcommand{\labelenumi}{\textbf{(\alph{enumi})}}

\begin{document}

\bigredheader{Homework \#2}

\vspace*{-1.25\bigskipamount}

\paragraph{By.} 
%
\textbf{Rui Xu(rx37), Yao Wang(yw438) \hfill \today}

\paragraph{Instructions.} 
%
This assignment may be completed with one partner. You and your
partner should submit a single solution on CMS. Please do not offer or
accept any other assistance on this assignment. Late submissions will
not be accepted.

\begin{exercise}
Prove the following theorem using the large-step semantics:

\begin{theorem*}
If $\sigma(i)$ is even and $\CONFIG{\sigma}{\while{b}{i := i+2}}
\stepsto \sigma'$ then $\sigma'(i)$ is also even.
\end{theorem*}
\noindent \textbf{Proof:} \textbf{CASE \emph{FALSE}:} When b is \emph{false}, 
\begin{mathpar}
	\inferrule*[left=WHILE-FALSE]
	{\CONFIG{\sigma}{$ \ b$} \stepsto False}
	{\CONFIG{\sigma}{\while{$b$}{$i := i + 2$}} \stepsto \sigma'}
\end{mathpar}
where $\sigma = \sigma'$. Since $\sigma(i)$ is even, $\sigma'(i)$ should also be even.\\

\noindent \textbf{CASE \emph{TRUE}:} When b is evaluated as \emph{true}, then\\
\begin{mathpar}
	\inferrule*[left=WHILE-T]
	{$\CONFIG{\sigma}{b}\stepsto True, \ \CONFIG{\sigma}{c}\stepsto$ \sigma', $ \ \CONFIG{\sigma'}{\while{b}{i : i+2}}\stepsto$ \sigma''}
	{$\CONFIG{\sigma}{\while{b}{i := i+2}}\stepsto$ \sigma''}
\end{mathpar}
Because $\sigma(i)$ is even and $\sigma'(i) = i + 2$, so $\sigma'(i)$ is even too. From induction hypothesis, if $\sigma'(i)$ is even and $\CONFIG{\sigma'}{\while{b}{i := i+2}}\stepsto \sigma''$ then $\sigma''(i)$ is also even. Therefore, $\sigma''(i)$ is also even. So theorem also holds for the case of true.
\end{exercise}

\begin{exercise}
Recall that IMP commands are equivalent if they always evaluate to the
same result:
\[
c_1 \sim c_2 ~~\defeq~~\forall \sigma,\sigma' \in \Set{Store}.~ <\sigma,c_1> \Downarrow \sigma' \Longleftrightarrow <\sigma,c_2> \Downarrow \sigma'.
\]
%
For each of the following pair of IMP commands, either use the
large-step operational semantics to prove they are equivalent or give
a concrete counter-example showing that they are not equivalent. You
may assume that the language has been extended with operators such as
$x != 0$.

\begin{enumerate}
\item 
$\assign{x}{a};~\assign{y}{a}$
\qquad and \qquad
$\assign{y}{a};~\assign{x}{a}$,\\
where $a$ is an arbitrary arithmetic expression

\noindent \textbf{Answer:} These two commands are not equivalent.\\
\noindent \textbf{Counter-example:} $a = x + y, \sigma(x) = 2, \sigma(y) = 3$. For the first command, $\CONFIG{\sigma}{x : = x + y; y : = x + y} \stepsto \sigma'$, where $\sigma'(x) = 5, \sigma'(y) = 8$. In terms of the second command, $\CONFIG{\sigma}{y: = x + y; x := x + y} \stepsto \sigma''$, where $\sigma''(x) = 7, \sigma''(y) = 5$. The final stores are obviously not equal, therefore, they are not equivalent commands.

\item 
$\WHILE~b~\DO~c$ 
\qquad and \qquad
$\IF~b~\THEN~(\WHILE~b~\DO~c); c~\ELSE~\SKIP$,\\
where $b$ is an arbitrary boolean expression and $c$ an arbitrary
command.

\noindent \textbf{Answer:} These two commands are not equivalent.\\
\noindent \textbf{Counter-example:} $\WHILE~i~\textless \ 2~\DO~i := i + 1$ \quad and \quad $\IF~i \ \textless \ 2~\THEN~(\WHILE~i \ \textless \ 2~\DO~i := i + 1); i := i + 1 \ \ELSE~\SKIP$, and $\sigma(i) = 0$. After applying the first command, $\sigma'(i) = 2$, while $\sigma''(i) = 3$ for the second command. Because the final stores are not equal, so they are not equivalent commands.

\item $\while{x~\mathord{!}\mathord{=}~0}{x := 0}$\qquad and \qquad $x:= 0 * x$

\noindent \textbf{Answer:} They are the equivalent commands.\\
\noindent \textbf{CASE $\sigma(x) = 0$:} then
\begin{mathpar}
	\inferrule*[left=WHILE-FALSE]
	{\CONFIG{\sigma}{$x != 0$} \stepsto FALSE}
	{\CONFIG{\sigma}{$\while{x~\mathord{!}\mathord{=}~0}{x := 0}$}\stepsto \sigma}
	
	\inferrule*[left=ASSGN]
	{\CONFIG{\sigma}{$0 * x$}\stepsto $0$}
	{\CONFIG{\sigma}{$x:= 0 * x$}\stepsto \sigma [$x$ \ \mapsto \ $0$]}
\end{mathpar}
\noindent Under the scenario where $\sigma(x) = 0$, the final stores are equal.\\

\noindent \textbf{CASE $\sigma(x) != 0$}:
\begin{mathpar}
	\inferrule*[left=WHILE-TRUE]
	{ \CONFIG{\sigma}{$x != 0$} \stepsto True, \
	\CONFIG{\sigma}{$x := 0$} \stepsto \sigma[$x$ \mapsto $0$], \
	\CONFIG{\sigma[$x$ \mapsto $0$]}{\while{$x != 0$}{$x := 0$}}\stepsto \sigma[$x$ \mapsto $0$]} 
	{\CONFIG{\sigma}{\while{$x != 0$}{$x := 0$}} \stepsto \sigma[$x$ \mapsto $0$]}
	
	\inferrule*[left=ASSGN]
	{\CONFIG{\sigma}{$0 * x$}\stepsto$0$}
	{\CONFIG{\sigma}{$x:= 0 * x$}\stepsto \sigma [$x$ \ \mapsto \ $0$]}
\end{mathpar}
\noindent If $x$ is not equal to zero, the final stores are still the same. For all possibilities, the results are the same. So these two commands are equivalent. 

\end{enumerate}
\end{exercise}

\begin{exercise}
Let $\CONFIG{\sigma}{c} \stepsone \CONFIG{\sigma'}{c'}$ be the
small-step operational semantics relation for IMP. Consider the
following definition of the multi-step relation:

\begin{mathpar}
\inferrule*[right=R1]{ 
}{
\CONFIG{\sigma}{c} \stepsone\kleenestar \CONFIG{\sigma}{c}
}

\inferrule*[right=R2]{
\CONFIG{\sigma}{c} \stepsone \CONFIG{\sigma'}{c'} \qquad
\CONFIG{\sigma'}{c'} \stepsone\kleenestar \CONFIG{\sigma''}{c''}
}{ \CONFIG{\sigma}{c} \stepsone\kleenestar \CONFIG{\sigma''}{c''} 
}
\end{mathpar}

\noindent Prove the following theorem, which states that
$\stepsone\kleenestar$ is transitive.

\begin{theorem*} If \( \CONFIG{\sigma}{c} \stepsone\kleenestar
\CONFIG{\sigma'}{c'} \) and \( \CONFIG{\sigma'}{c'}
\stepsone\kleenestar \CONFIG{\sigma''}{c''} \) then \(
\CONFIG{\sigma}{c} \stepsone\kleenestar \CONFIG{\sigma''}{c''} \).
\end{theorem*}

\noindent \textbf{Proof:} \textbf{CASE 1:} If $\CONFIG{\sigma}{c}$ is equal to $\CONFIG{\sigma'}{c'}$, since $\CONFIG{\sigma'}{c'} \stepsone\kleenestar \CONFIG{\sigma''}{c''}$, so we get that $\CONFIG{\sigma}{c} \stepsone\kleenestar \CONFIG{\sigma''}{c''}$\\

\noindent \textbf{CASE 2:} If $\CONFIG{\sigma'}{c'}$ is equal to $\CONFIG{\sigma''}{c''}$, since $\CONFIG{\sigma}{c}$ \stepsone\kleenestar $\CONFIG{\sigma'}{c'}$ so we get that $\CONFIG{\sigma}{c} \stepsone\kleenestar \CONFIG{\sigma''}{c''}$\\

\noindent \textbf{CASE 3:} If these three combinations of store and commands are not equal, from the \textbf{\emph{R2}} relation, we can get that there exists $\CONFIG{\sigma'''}{c'''}$ having the characteristic:
\begin{mathpar}
	\inferrule*[right=R2]
	{ \CONFIG{\sigma}{$c$} \stepsone \CONFIG{\sigma'''}{$c'''$} \quad
	  \CONFIG{\sigma'''}{$c'''$} \stepsone\kleenestar \CONFIG{\sigma'}{$c'$} }
	{ \CONFIG{\sigma}{c} \stepsone\kleenestar \CONFIG{\sigma'}{$c'$} }
\end{mathpar}
From the induction hypothesis, if $\CONFIG{\sigma'''}{c'''}$ \stepsone\kleenestar $\CONFIG{\sigma'}{c'}$ and $\CONFIG{\sigma'}{c'}$ \stepsone\kleenestar $\CONFIG{\sigma''}{c''}$, there should be $\CONFIG{\sigma'''}{c'''}$ \stepsone\kleenestar $\CONFIG{\sigma''}{c''}$. Then we use \textbf{\emph{R2}} relationship:
\begin{mathpar}
	\inferrule*[right=R2]
	{ \CONFIG{\sigma}{$c$} \stepsone \CONFIG{\sigma'''}{$c'''$} \quad
	  \CONFIG{\sigma'''}{$c'''$} \stepsone \CONFIG{\sigma''}{$c''$}}
	{ \CONFIG{\sigma}{$c$} \stepsone\kleenestar \CONFIG{\sigma''}{$c''$}}
\end{mathpar}
So from $\CONFIG{\sigma}{c}$ \stepsone\kleenestar $\CONFIG{\sigma'}{c'}$ and $\CONFIG{\sigma'}{c'}$ \stepsone\kleenestar $\CONFIG{\sigma''}{c''}$, we can get that \CONFIG{\sigma}{$c$} \stepsone\kleenestar \CONFIG{\sigma''}{$c''$}
\end{exercise}

\begin{exercise}

\newcommand{\THROW}[1]{\ensuremath{\impfnt{throw}~#1}}
\newcommand{\TRYCATCH}[3]{\ensuremath{\impfnt{try}~#1~\impfnt{with}~#2~\DO~#3}}

In this exercise, you will extend the IMP language with
exceptions. These exceptions are intended to behave like the analogous
constructs found in languages such as Java. We will proceed in several
steps.

First, we fix a set of exceptions, which will ranged over by
metavariables $e$, and we extend the syntax of the language with new
commands for throwing and handling exceptions:
%
\[
\begin{array}{r@{~}c@{~}l}
 c & ::= & \SKIP \\
& \mid & x := a\\
& \mid & c_1 ; c_2\\
& \mid & \cond{b}{c_1}{c_2} \\
& \mid & \while{b}{c} \\
& \mid & \shade{\THROW{e}} \\
& \mid & \shade{\TRYCATCH{c_1}{e}{c_2}}
\end{array}
\]
%
Intuitively, evaluating a command either yields a modified store or a
pair comprising a modified store and an (uncaught) exception. We let
metavariables $r$ range over such results:
%
\[
r ::= \sigma \mid (\sigma,e)
\]
%

Second, we change the type of the large-step evaluation relation so it
yields a result instead of a store: $<\sigma,c> \stepsto r$.

Third, we will extend the large-step semantics rules so they handle
\impfnt{throw} and \impfnt{try} commands. This is your task in this
exercise. 

Informally, $\THROW~e$ should return exception $e$, and
$\TRYCATCH{c_1}{e}{c_2}$ should execute $c_1$ and return the result it
produces, unless the result contains an exception $e$, in which case
it should discard $e$ executes the handler $c_2$. You will also need
to modify many other rules so they have the right type and also
propagate exceptions.\\

\noindent \textbf{Answer: }
\begin{mathpar}
	\inferrule*[left=SKIP]
	{ }
	{\CONFIG{\sigma}{SKIP}\stepsto \sigma}

	\inferrule*[left=ASSGN]
	{\CONFIG{\sigma}{$a$}\stepsto \ $n$}
	{\CONFIG{\sigma}{$ \ x := a$}\stepsto \sigma[$x$ \ \mapsto \ $n$]}
	
	\inferrule*[left=SEQ1]
	{ {\CONFIG{\sigma}{c_1} \stepsto \sigma'}, {\CONFIG{\sigma'}{c_2} \stepsto \sigma''}}
	{\CONFIG{\sigma}{c_1 ; c_2} \stepsto \sigma''}
	
	\inferrule*[left=SEQ2]
	{ {\CONFIG{\sigma}{c_1} \stepsto \sigma'}, {\CONFIG{\sigma'}{c_2} \stepsto (\sigma'',e)} }
	{\CONFIG{\sigma}{c_1 ; c_2} \stepsto (\sigma'',e)}
	

	\inferrule*[left=SEQ3]
	{ \CONFIG{\sigma}{c_1} \stepsto (\sigma',e)}
	{ \CONFIG{\sigma}{c_1 ; c_2} \stepsto (\sigma', e)}
	
\end{mathpar}

\begin{mathpar}
	\inferrule*[left=IF-T1]
	{ \CONFIG{\sigma}{$ \ b$} \stepsto True, \ \CONFIG{\sigma}{c_1} \stepsto \sigma'}
	{ \CONFIG{\sigma}{\cond{$b$}{c_1}{c_2}} \stepsto \sigma'}
	
	\inferrule*[left=IF-T2]
	{ \CONFIG{\sigma}{$ \ b$} \stepsto True, \ \CONFIG{\sigma}{c_1} \stepsto (\sigma', e)}
	{ \CONFIG{\sigma}{\cond{$b$}{c_1}{c2}} \stepsto (\sigma', e)}
	
	\inferrule*[left=IF-F1]
	{ \CONFIG{\sigma}{$ \ b$} \stepsto False, \ \CONFIG{\sigma}{c_2} \stepsto \sigma''}
	{ \CONFIG{\sigma}{\cond{$b$}{c_1}{c_2}} \stepsto \sigma''}
	
	\inferrule*[left=IF-T2]
	{ \CONFIG{\sigma}{$b$} \stepsto False, \ \CONFIG{\sigma}{c_2} \stepsto (\sigma'', e)}
	{ \CONFIG{\sigma}{\cond{$b$}{c_1}{c2}} \stepsto (\sigma'', e)}
	
	\inferrule*[left=WHILE-F]
	{ \CONFIG{\sigma}{$ \ b$} \stepsto False }
	{ \CONFIG{\sigma}{\while{$b$}{$c$}} \stepsto \sigma}
	
	\inferrule*[left=WHILE-T1]
	{ \CONFIG{\sigma}{$ \ b$} \stepsto True,
	  \CONFIG{\sigma}{$ \ c$} \stepsto \sigma',
	  \CONFIG{\sigma'}{\while{$b$}{$c$}} \stepsto \sigma''
	}
	{ \CONFIG{\sigma}{\while{$b$}{$c$}} \stepsto \sigma''}
	
	\inferrule*[left=WHILE-T2]
	{ \CONFIG{\sigma}{$ \ b$} \stepsto True,
	  \CONFIG{\sigma}{$ \ c$} \stepsto \sigma',
	  \CONFIG{\sigma'}{\while{$b$}{$c$}} \stepsto (\sigma'', e)
	}
	{ \CONFIG{\sigma}{\while{$b$}{$c$}} \stepsto (\sigma'', e)}
	
	\inferrule*[left=WHILE-T3]
	{ \CONFIG{\sigma}{$ \ b$} \stepsto True,
	  \CONFIG{\sigma}{$ \ c$} \stepsto (\sigma', e)
	}
	{ \CONFIG{\sigma}{\while{$b$}{$c$}} \stepsto (\sigma', e)}
	
	\inferrule*[left=THROW]
	{ }
	{\CONFIG{\sigma}{\THROW{$e$}} \stepsto (\sigma, $e$)}
	
	\inferrule*[left=TRY1]
	{ \CONFIG{\sigma}{c_1} \stepsto \sigma'}
	{ \CONFIG{\sigma}{\TRYCATCH{c_1}{e}{c_2}} \stepsto \sigma'}
	
	\inferrule*[left=TRY2(where $e_1 \neq e_2$)]
	{ \CONFIG{\sigma}{c_1} \stepsto (\sigma', e_2)}
	{ \CONFIG{\sigma}{\TRYCATCH{c_1}{e_1}{c_2}} \stepsto (\sigma', e_2)}
	
	\inferrule*[left=TRY3]
	{ \CONFIG{\sigma}{c_1} \stepsto (\sigma', e), \CONFIG{\sigma'}{c_2} \stepsto \sigma''}
	{ \CONFIG{\sigma}{\TRYCATCH{c_1}{e}{c_2}} \stepsto \sigma''}
	
	\inferrule*[left=TRY4]
	{ \CONFIG{\sigma}{c_1} \stepsto (\sigma', e), \CONFIG{\sigma'}{c_2} \stepsto (\sigma'', e')}
	{ \CONFIG{\sigma}{\TRYCATCH{c_1}{e}{c_2}} \stepsto (\sigma'',e')}
	
\end{mathpar}

\end{exercise}

\begin{debriefing} \hfill\\[-4ex]
\begin{enumerate*}
\item How many hours did you spend on this assignment?\\
\noindent \textbf{Answer:} Five hours\\
\item Would you rate it as easy, moderate, or difficult? \\
\noindent \textbf{Answer:} Moderate\\
\item How deeply do you feel you understand the material it covers (0\%–100\%)? \\
\noindent \textbf{Answer:} About 85\% \\
\item If you have any other comments, we would like to hear them!
  Please write them here or send email to
  \mtt{jnfoster@cs.cornell.edu}.
\end{enumerate*}
\end{debriefing}
\end{document}

